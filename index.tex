% Options for packages loaded elsewhere
\PassOptionsToPackage{unicode}{hyperref}
\PassOptionsToPackage{hyphens}{url}
\PassOptionsToPackage{dvipsnames,svgnames,x11names}{xcolor}
%
\documentclass[
]{agujournal2019}

\usepackage{amsmath,amssymb}
\usepackage{iftex}
\ifPDFTeX
  \usepackage[T1]{fontenc}
  \usepackage[utf8]{inputenc}
  \usepackage{textcomp} % provide euro and other symbols
\else % if luatex or xetex
  \usepackage{unicode-math}
  \defaultfontfeatures{Scale=MatchLowercase}
  \defaultfontfeatures[\rmfamily]{Ligatures=TeX,Scale=1}
\fi
\usepackage{lmodern}
\ifPDFTeX\else  
    % xetex/luatex font selection
\fi
% Use upquote if available, for straight quotes in verbatim environments
\IfFileExists{upquote.sty}{\usepackage{upquote}}{}
\IfFileExists{microtype.sty}{% use microtype if available
  \usepackage[]{microtype}
  \UseMicrotypeSet[protrusion]{basicmath} % disable protrusion for tt fonts
}{}
\makeatletter
\@ifundefined{KOMAClassName}{% if non-KOMA class
  \IfFileExists{parskip.sty}{%
    \usepackage{parskip}
  }{% else
    \setlength{\parindent}{0pt}
    \setlength{\parskip}{6pt plus 2pt minus 1pt}}
}{% if KOMA class
  \KOMAoptions{parskip=half}}
\makeatother
\usepackage{xcolor}
\setlength{\emergencystretch}{3em} % prevent overfull lines
\setcounter{secnumdepth}{5}
% Make \paragraph and \subparagraph free-standing
\makeatletter
\ifx\paragraph\undefined\else
  \let\oldparagraph\paragraph
  \renewcommand{\paragraph}{
    \@ifstar
      \xxxParagraphStar
      \xxxParagraphNoStar
  }
  \newcommand{\xxxParagraphStar}[1]{\oldparagraph*{#1}\mbox{}}
  \newcommand{\xxxParagraphNoStar}[1]{\oldparagraph{#1}\mbox{}}
\fi
\ifx\subparagraph\undefined\else
  \let\oldsubparagraph\subparagraph
  \renewcommand{\subparagraph}{
    \@ifstar
      \xxxSubParagraphStar
      \xxxSubParagraphNoStar
  }
  \newcommand{\xxxSubParagraphStar}[1]{\oldsubparagraph*{#1}\mbox{}}
  \newcommand{\xxxSubParagraphNoStar}[1]{\oldsubparagraph{#1}\mbox{}}
\fi
\makeatother


\providecommand{\tightlist}{%
  \setlength{\itemsep}{0pt}\setlength{\parskip}{0pt}}\usepackage{longtable,booktabs,array}
\usepackage{calc} % for calculating minipage widths
% Correct order of tables after \paragraph or \subparagraph
\usepackage{etoolbox}
\makeatletter
\patchcmd\longtable{\par}{\if@noskipsec\mbox{}\fi\par}{}{}
\makeatother
% Allow footnotes in longtable head/foot
\IfFileExists{footnotehyper.sty}{\usepackage{footnotehyper}}{\usepackage{footnote}}
\makesavenoteenv{longtable}
\usepackage{graphicx}
\makeatletter
\def\maxwidth{\ifdim\Gin@nat@width>\linewidth\linewidth\else\Gin@nat@width\fi}
\def\maxheight{\ifdim\Gin@nat@height>\textheight\textheight\else\Gin@nat@height\fi}
\makeatother
% Scale images if necessary, so that they will not overflow the page
% margins by default, and it is still possible to overwrite the defaults
% using explicit options in \includegraphics[width, height, ...]{}
\setkeys{Gin}{width=\maxwidth,height=\maxheight,keepaspectratio}
% Set default figure placement to htbp
\makeatletter
\def\fps@figure{htbp}
\makeatother
% definitions for citeproc citations
\NewDocumentCommand\citeproctext{}{}
\NewDocumentCommand\citeproc{mm}{%
  \begingroup\def\citeproctext{#2}\cite{#1}\endgroup}
\makeatletter
 % allow citations to break across lines
 \let\@cite@ofmt\@firstofone
 % avoid brackets around text for \cite:
 \def\@biblabel#1{}
 \def\@cite#1#2{{#1\if@tempswa , #2\fi}}
\makeatother
\newlength{\cslhangindent}
\setlength{\cslhangindent}{1.5em}
\newlength{\csllabelwidth}
\setlength{\csllabelwidth}{3em}
\newenvironment{CSLReferences}[2] % #1 hanging-indent, #2 entry-spacing
 {\begin{list}{}{%
  \setlength{\itemindent}{0pt}
  \setlength{\leftmargin}{0pt}
  \setlength{\parsep}{0pt}
  % turn on hanging indent if param 1 is 1
  \ifodd #1
   \setlength{\leftmargin}{\cslhangindent}
   \setlength{\itemindent}{-1\cslhangindent}
  \fi
  % set entry spacing
  \setlength{\itemsep}{#2\baselineskip}}}
 {\end{list}}
\usepackage{calc}
\newcommand{\CSLBlock}[1]{\hfill\break\parbox[t]{\linewidth}{\strut\ignorespaces#1\strut}}
\newcommand{\CSLLeftMargin}[1]{\parbox[t]{\csllabelwidth}{\strut#1\strut}}
\newcommand{\CSLRightInline}[1]{\parbox[t]{\linewidth - \csllabelwidth}{\strut#1\strut}}
\newcommand{\CSLIndent}[1]{\hspace{\cslhangindent}#1}

\usepackage{url} %this package should fix any errors with URLs in refs.
\usepackage{lineno}
\usepackage[inline]{trackchanges} %for better track changes. finalnew option will compile document with changes incorporated.
\usepackage{soul}
\linenumbers
\makeatletter
\@ifpackageloaded{caption}{}{\usepackage{caption}}
\AtBeginDocument{%
\ifdefined\contentsname
  \renewcommand*\contentsname{Table of contents}
\else
  \newcommand\contentsname{Table of contents}
\fi
\ifdefined\listfigurename
  \renewcommand*\listfigurename{List of Figures}
\else
  \newcommand\listfigurename{List of Figures}
\fi
\ifdefined\listtablename
  \renewcommand*\listtablename{List of Tables}
\else
  \newcommand\listtablename{List of Tables}
\fi
\ifdefined\figurename
  \renewcommand*\figurename{Figure}
\else
  \newcommand\figurename{Figure}
\fi
\ifdefined\tablename
  \renewcommand*\tablename{Table}
\else
  \newcommand\tablename{Table}
\fi
}
\@ifpackageloaded{float}{}{\usepackage{float}}
\floatstyle{ruled}
\@ifundefined{c@chapter}{\newfloat{codelisting}{h}{lop}}{\newfloat{codelisting}{h}{lop}[chapter]}
\floatname{codelisting}{Listing}
\newcommand*\listoflistings{\listof{codelisting}{List of Listings}}
\makeatother
\makeatletter
\makeatother
\makeatletter
\@ifpackageloaded{caption}{}{\usepackage{caption}}
\@ifpackageloaded{subcaption}{}{\usepackage{subcaption}}
\makeatother

\ifLuaTeX
  \usepackage{selnolig}  % disable illegal ligatures
\fi
\usepackage{bookmark}

\IfFileExists{xurl.sty}{\usepackage{xurl}}{} % add URL line breaks if available
\urlstyle{same} % disable monospaced font for URLs
\hypersetup{
  pdftitle={Project thesis},
  pdfauthor={Radin Rahimi; Philipp Pelz},
  pdfkeywords={Strain Mapping, py4DSTEM},
  colorlinks=true,
  linkcolor={blue},
  filecolor={Maroon},
  citecolor={Blue},
  urlcolor={Blue},
  pdfcreator={LaTeX via pandoc}}



\draftfalse

\begin{document}
\title{Project thesis}

\authors{Radin Rahimi\affil{1}, Philipp Pelz\affil{1}}
\affiliation{1}{Curvenote, }
\correspondingauthor{Radin Rahimi}{radin.rahimi@fau.de}







\section{Introduction}\label{introduction}

\textsubscript{Source:
\href{https://radinrahimi1998.github.io/project-thesis/index.ipynb.html}{Article
Notebook}}

In semiconductor devices to estimate intended and unintended strain
distributions it is vitally important to use Measurtement of strain with
high spatial resolution and high preciscion (Zeltmann et al., 2020). It
is also neccesary have combination of high resolution with large field
of view.

\textsubscript{Source:
\href{https://radinrahimi1998.github.io/project-thesis/index.ipynb.html}{Article
Notebook}}

X-rays techniques are able measure strain with points that it is
important, high resolution, high precision(≈10\^{}(-5)(Darbal et al.,
2013) (Robinson \& Harder, 2009))and large field of view, but they do
not keep the need of high spatial resolution(≈500nm(Darbal et al.,
2013)). Which makes them inappropriate for analyzing the next generation
of nanoscale materials and devices (Darbal et al., 2013) (Robinson \&
Harder, 2009).

\textsubscript{Source:
\href{https://radinrahimi1998.github.io/project-thesis/index.ipynb.html}{Article
Notebook}}

Quantitive structure retrieval using computer-controlled high resolution
electron microscopy(HREM)images is much less frequent rather than X-ray
diffraction patterns, despite its potential for applications including
interfaces and dislocations in area like high localized crystal defect
structures (Möbus et al., 1998) . These methods is dictated the local
intensity in micrograph by the position of atomic columns (Du \&
Phillipp, 2006) . Using images instead of diffraction offers high
spatial resolution, but the field of view is limited (Mahr et al., 2015)
. The shape of the measured lattice strain profiles can indicate
artifacts depending on the selected imaging conditions. These artifacts
arise from continuous contrast variations extending across several
monolayers from the interface positions, caused by local crystal tilts
in elastically relaxed specimens. Despite this, the average strain in
thicker layers can be measured with adequate accuracy, providing a rough
estimate of layer compositions when analyzing experimental micrographs
(Tillmann et al., 2000) .

\textsubscript{Source:
\href{https://radinrahimi1998.github.io/project-thesis/index.ipynb.html}{Article
Notebook}}

On the other hand, strain can be measured using TEM techniques based on
diffraction. This paper discusses six different TEM techniques which
provide the best spatial resolution (below 5 nm(Zeltmann et al., 2020))
and quantitative strain measurements in the TEM (Darbal et al., 2013)
(Robinson \& Harder, 2009). NBED, CBED, HRTEM, DFEH, HRSTEM, HOLZ.

\textsubscript{Source:
\href{https://radinrahimi1998.github.io/project-thesis/index.ipynb.html}{Article
Notebook}}

\textsubscript{Source:
\href{https://radinrahimi1998.github.io/project-thesis/index.ipynb.html}{Article
Notebook}}

\section{Methods}\label{sec-methods}

\textsubscript{Source:
\href{https://radinrahimi1998.github.io/project-thesis/index.ipynb.html}{Article
Notebook}}

\subsection{NBED}\label{nbed}

\textsubscript{Source:
\href{https://radinrahimi1998.github.io/project-thesis/index.ipynb.html}{Article
Notebook}}

\section{Conclusion}\label{conclusion}

\textsubscript{Source:
\href{https://radinrahimi1998.github.io/project-thesis/index.ipynb.html}{Article
Notebook}}

\section*{References}\label{references}
\addcontentsline{toc}{section}{References}

\vspace{1em}

\textsubscript{Source:
\href{https://radinrahimi1998.github.io/project-thesis/index.ipynb.html}{Article
Notebook}}

\phantomsection\label{refs}
\begin{CSLReferences}{1}{0}
\bibitem[\citeproctext]{ref-darbal_automated_2013}
Darbal, A. D., Narayan, R. D., Vartuli, C., Lian, G., Graham, R.,
Shaapur, F., et al. (2013). Automated high precision strain measurement
using nanobeam diffraction coupled with precession. \emph{Microscopy and
Microanalysis}, \emph{19}, 702--703.
\url{https://doi.org/10.1017/S1431927613005503}

\bibitem[\citeproctext]{ref-du_accuracy_2006}
Du, K., \& Phillipp, F. (2006). On the accuracy of lattice‐distortion
analysis directly from high‐resolution transmission electron
micrographs. \emph{Journal of Microscopy}, \emph{221}(1), 63--71.
\url{https://doi.org/10.1111/j.1365-2818.2006.01536.x}

\bibitem[\citeproctext]{ref-mahr_theoretical_2015}
Mahr, C., Müller-Caspary, K., Grieb, T., Schowalter, M., Mehrtens, T.,
Krause, F. F., et al. (2015). Theoretical study of precision and
accuracy of strain analysis by nano-beam electron diffraction.
\emph{Ultramicroscopy}, \emph{158}, 38--48.
\url{https://doi.org/10.1016/j.ultramic.2015.06.011}

\bibitem[\citeproctext]{ref-mobus_iterative_1998}
Möbus, G., Schweinfest, R., Gemming, T., Wagner, T., \& Rühle, M.
(1998). Iterative structure retrieval techniques in {HREM}: A
comparative study and a modular program package. \emph{Journal of
Microscopy}, \emph{190}(1), 109--130.
\url{https://doi.org/10.1046/j.1365-2818.1998.3120865.x}

\bibitem[\citeproctext]{ref-robinson_coherent_2009}
Robinson, I., \& Harder, R. (2009). Coherent x-ray diffraction imaging
of strain at the nanoscale. \emph{Nature Materials}, \emph{8}(4),
291--298. \url{https://doi.org/10.1038/nmat2400}

\bibitem[\citeproctext]{ref-tillmann_impact_2000}
Tillmann, K., Lentzen, M., \& Rosenfeld, R. (2000). Impact of column
bending in high-resolution transmission electron microscopy on the
strain evaluation of {GaAs}/{InAs}/{GaAs} heterostructures.
\emph{Ultramicroscopy}, \emph{83}(1), 111--128.
\url{https://doi.org/10.1016/S0304-3991(99)00175-8}

\bibitem[\citeproctext]{ref-zeltmann_patterned_2020}
Zeltmann, S. E., Müller, A., Bustillo, K. C., Savitzky, B., Hughes, L.,
Minor, A. M., \& Ophus, C. (2020). Patterned probes for high precision
4D-{STEM} bragg measurements. \emph{Ultramicroscopy}, \emph{209},
112890. \url{https://doi.org/10.1016/j.ultramic.2019.112890}

\end{CSLReferences}




\end{document}
